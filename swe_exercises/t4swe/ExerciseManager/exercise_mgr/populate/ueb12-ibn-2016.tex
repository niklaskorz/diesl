%% LyX 2.3.4.4 created this file.  For more info, see http://www.lyx.org/.
%% Do not edit unless you really know what you are doing.
\documentclass[11pt,a4paper,ngerman,headinclude,footinclude,DIV14,smallheadings,ngerman]{scrartcl}
\usepackage[T1]{fontenc}
\usepackage[utf8]{inputenc}
\setlength{\parskip}{\medskipamount}
\setlength{\parindent}{0pt}
\usepackage{babel}
\usepackage{verbatim}
\usepackage{calc}
\usepackage{graphicx}
\usepackage[unicode=true]
 {hyperref}

\makeatletter

%%%%%%%%%%%%%%%%%%%%%%%%%%%%%% LyX specific LaTeX commands.
\pdfpageheight\paperheight
\pdfpagewidth\paperwidth


%%%%%%%%%%%%%%%%%%%%%%%%%%%%%% User specified LaTeX commands.
%\usepackage[ngerman]{babel}
\usepackage{xspace}
\usepackage{caption}
\usepackage[autolanguage]{numprint}
\usepackage{booktabs}
\usepackage{tikz}
\usepackage{ifthen}
\usepackage{paralist}
\usepackage{scrpage2}
\usepackage[a4paper]{geometry}
\geometry{bmargin=4.5cm}
%\usepackage{libertine} 

\newcommand{\uebnr}{11}
\newcommand{\uebabgabe}{19.07.2016, 12:00 Uhr}

\newcommand{\uebdatum}{12.07.2016}
 
\newcounter{uebnummer}
\setcounter{uebnummer}{0}

\newcommand{\aufgabe}[1]{\stepcounter{uebnummer}\section*{Aufgabe \arabic{uebnummer} \hfill(#1 \ifthenelse{\equal{#1}{1}}{Punkt}{Punkte})}}

\pagestyle{scrheadings}

\ifoot{\footnotesize{\uebnr{}. Übung IBN -- SoSe 2016}}
\cfoot{}
\ofoot{\thepage}

%\newcommand{\befehl}[1]{"`\texttt{#1}"'\xspace}
\setlength{\parindent}{0pt} 

% for quotations:
\makeatletter
\newenvironment{chapquote}[2][2em]
  {\setlength{\@tempdima}{#1}%
   \def\chapquote@author{#2}%
   \parshape 1 \@tempdima \dimexpr\textwidth-2\@tempdima\relax%
   \itshape}
  {\par\normalfont\hfill--\ \chapquote@author\hspace*{\@tempdima}\par\bigskip}
\makeatother

\makeatother

\begin{document}
Universität Heidelberg \hfill{}Lehrstuhl für Parallele und Verteilte
Systeme (PVS)\\
 Sommersemester 2016 \hfill{}Artur Andrzejak, Lutz Büch

\begin{center}
\textbf{\large{}\uebnr{}. Übung zur Vorlesung ,,Betriebssysteme
und Netzwerke`` (IBN)\\}\vspace*{2ex}
Abgabedatum: \uebabgabe\textbf{\large{}\\}\rule{1\columnwidth}{0.5pt}
\par\end{center}

\begin{comment}
Router, DHCP, NAT, ICMP, Routingalgorithmen

DHCP, NAT, ICMP evtl. auf dem nächsten Zettel.
\end{comment}

\aufgabe{Bonus, 2}Die FIFA/UEFA Fußball-Welt- und Europameisterschaften
brechen regelmäßig Rekorde beim Streamingverkehr (mehrere Terabits
pro Sekunde)\footnote{\href{https://labs.ripe.net/Members/fergalc/internet-traffic-during-euro-2012-final-conclusions}{https://labs.ripe.net/Members/fergalc/internet-traffic-during-euro-2012-final-conclusions}\href{http://variety.com/2014/digital/news/world-cup-sets-new-internet-video-streaming-record-1201221997/}{http://variety.com/2014/digital/news/world-cup-sets-new-internet-video-streaming-record-1201221997/}\href{http://advanced-television.com/2016/06/20/euro-2016-record-streaming-traffic/}{http://advanced-television.com/2016/06/20/euro-2016-record-streaming-traffic/}}.
Solche enormen Durchsätze lassen sich nicht einem einfachen Server-Client-Modell
realisieren. Recherchieren und erklären Sie, was Content Delivery
Networks sind, wie Request Routing funktioniert und wie daran DNS
beteiligt ist.%
\begin{comment}
http://de.wikipedia.org/wiki/Content\_Delivery\_Network\#Funktionsweise
\end{comment}

\aufgabe{2}Verbinden Sie sich mit einem Rechner mit dem Uninetz.
\begin{description}
\item [{a)}] Was ist ihre IP-Adresse? Sind Sie über ein NAT mit dem Internet
verbunden? Was ist der Netzpräfix? Wie viele Hosts können sich maximal
in diesem Subnetz anschließen? Geben Sie an, wie Sie diese Daten herausgefunden
haben.
\item [{b)}] Recherchieren Sie, was ein Smurf-Angriff\footnote{\href{http://de.wikipedia.org/wiki/Smurf-Angriff}{http://de.wikipedia.org/wiki/Smurf-Angriff}}
ist. Erklären Sie mit eigenen Worten das Prinzip und was es mit ICMP
zu tun hat. Warum funktioniert ein solcher Angriff heute in der Regel
nicht mehr?%
\begin{comment}
Zur letzten Frage: http://en.wikipedia.org/wiki/Smurf\_attack\#History\\
und http://en.wikipedia.org/wiki/Smurf\_attack\#Mitigation
\end{comment}
\begin{comment}
\#14-12-1
\end{comment}
\end{description}
\medskip{}

\fbox{\begin{minipage}[t]{1\columnwidth - 2\fboxsep - 2\fboxrule}%
\begin{chapquote}{Leslie Lamport, \textit{\LaTeX, Lamport-Uhr}} A distributed system is one in which the failure of a computer you didn't even know existed can render your own computer unusable.\end{chapquote}%
\end{minipage}}

\aufgabe{2}Betrachten Sie die Netzkonfiguration aus der folgenden
Abbildung (Kurose et al., Abbildung 4.22, Seite 390). Nehmen Sie an,
dass der ISP dem Router die Adresse 126.13.89.67 zuweist und dass
die Netzadresse des Heimnetzes 192.168/16 ist. 
\begin{description}
\item [{a)}] Weisen Sie allen Schnittstellen im Heimnetz Adressen zu.
\item [{b)}] Nehmen Sie an, dass jeder Host zwei laufende TCP-Verbindungen
hat, die alle mit Port 80 auf Host 128.119.40.86 verbunden sind. Erstellen
Sie die sechs entsprechenden Einträge in der NAT-Übersetzungstabelle.
\end{description}
\begin{center}
\includegraphics[scale=0.3,bb = 0 0 200 100, draft, type=eps]{NAT.pdf}
\par\end{center}

\begin{comment}
\begin{figure}
\caption{\protect\includegraphics[bb = 0 0 200 100, draft, type=eps]{../../../Material/IBN/Übungssammlung/1011-14-3.jpg}}
\end{figure}
\end{comment}
\begin{comment}
\#10/11-14-3 \#13-9-9
\end{comment}
\begin{comment}
Kurose Kap 4, Problem 18 a) Home addresses: 192.168.0.1, 192.168.0.2,
192.168.0.3 with the router interface being 192.168.0.4 b) NAT Translation
Table WAN Side LAN Side 128.119.40.86, 4000 192.168.0.1, 3345 128.119.40.86,
4001 192.168.0.1, 3346 128.119.40.86, 4002 192.168.0.2, 3445 128.119.40.86,
4003 192.168.0.2, 3446 128.119.40.86, 4004 192.168.0.3, 3545 128.119.40.86,
4005 192.168.0.3, 3546 \\

Silvestre Zabala, 18. Juli 2013:

Nach IBN 07 Folie 31 und Kurose (5. Auflage) Seite 360 steht in der
NAT-Tabelle die Quelladresse des Routers.
\end{comment}

\aufgabe{2}Mit dem Tool j-Algo\footnote{\href{http://j-algo.binaervarianz.de/}{http://j-algo.binaervarianz.de/}}
können Sie verschiedene Algorithmen visualisieren. Führen Sie den
Dijkstra-Algorithmus an dem Beispielgraphen aus, der mit j-Algo geliefert
wird (\emph{examples\textbackslash dijkstra\textbackslash vorlesungsgraph.jalgo}).
\begin{itemize}
\item Überlegen Sie sich allgemein: Wie viele grüne Kanten wird es geben,
wenn man einen verbundenen Graphen mit $n$ Knoten eingibt?%
\begin{comment}
Es werden immer n-1 Kanten sein, da die grünen Kanten einen Spannbaum
repräsentieren.
\end{comment}
\item Ergeben die grün markierten Kanten (immer) einen minimalen Spannbaum?
Falls ja, begründen Sie dies. Falls nein, geben Sie ein Gegenbeispiel.
\begin{comment}
Nein, als Gegenbeispiel betrachte man folgenden Graphen (Notation:
(Knoten1, Kantengewicht, Knoten2)):

(1,2,2), (1,2,3), (1,2,4), (2,1,3), (3,1,4)

Der Baum, der bei Dijkstra entsteht, wenn man als Startknoten 1 auswählt,
hat Gewicht 6, während ein Minimaler Spannbaum Gewicht 4 hat.
\end{comment}
\end{itemize}
\aufgabe{3}Betrachten Sie die Netzwerktopologie mit drei Knoten aus
dem Beispiel zum Distanzvektor-Routing-Algorithmus aus der Vorlesung
N07. Statt den dort angegebenen Kantengewichten sollen nun folgende
gelten: c(x,y)=5, c(y,z)=6 und c(z,x)=2. Berechnen Sie die Entfernungstabellen
nach dem Initialisierungsschritt und nach jeder Iteration einer synchronen
Version des Distanzvektor-Algorithmus (analog zum genannten Beispiel
aus Vorlesung N08). Dann nehmen Sie eine Änderung des Kantengewichts
c(y,z) auf 8 vor. Wie lange dauert es, bis die Knoten ihre Tabellen
auf den neuesten Stand gebracht haben?%
\begin{comment}
x

\begin{tabular}{|c|c|c|c|}
\hline 
 & \textbf{x} & \textbf{y} & \textbf{z}\tabularnewline
\hline 
\hline 
\textbf{x} & 0 & 5 & 2\tabularnewline
\hline 
\textbf{y} & \ensuremath{\infty} & \ensuremath{\infty} & \ensuremath{\infty}\tabularnewline
\hline 
\textbf{z} & \ensuremath{\infty} & \ensuremath{\infty} & \ensuremath{\infty}\tabularnewline
\hline 
\end{tabular} =>%
\begin{tabular}{|c|c|c|c|}
\hline 
 & \textbf{x} & \textbf{y} & \textbf{z}\tabularnewline
\hline 
\hline 
\textbf{x} & 0 & 5 & 2\tabularnewline
\hline 
\textbf{y} & 5 & 0 & 6\tabularnewline
\hline 
\textbf{z} & 2 & 6 & 0\tabularnewline
\hline 
\end{tabular}=> Änderung zu c(y,z)=8 =>%
\begin{tabular}{|c|c|c|c|}
\hline 
 & \textbf{x} & \textbf{y} & \textbf{z}\tabularnewline
\hline 
\hline 
\textbf{x} & 0 & 5 & 2\tabularnewline
\hline 
\textbf{y} & 5 & 0 & 8\tabularnewline
\hline 
\textbf{z} & 2 & 8 & 0\tabularnewline
\hline 
\end{tabular}=>%
\begin{tabular}{|c|c|c|c|}
\hline 
 & \textbf{x} & \textbf{y} & \textbf{z}\tabularnewline
\hline 
\hline 
\textbf{x} & 0 & 5 & 2\tabularnewline
\hline 
\textbf{y} & 5 & 0 & 8\tabularnewline
\hline 
\textbf{z} & 2 & 8 & 0\tabularnewline
\hline 
\end{tabular}=>%
\begin{tabular}{|c|c|c|c|}
\hline 
 & \textbf{x} & \textbf{y} & \textbf{z}\tabularnewline
\hline 
\hline 
\textbf{x} & 0 & 5 & 2\tabularnewline
\hline 
\textbf{y} & 5 & 0 & 7\tabularnewline
\hline 
\textbf{z} & 2 & 7 & 0\tabularnewline
\hline 
\end{tabular}

y

\begin{tabular}{|c|c|c|c|}
\hline 
 & \textbf{x} & \textbf{y} & \textbf{z}\tabularnewline
\hline 
\hline 
\textbf{x} & \ensuremath{\infty} & \ensuremath{\infty} & \ensuremath{\infty}\tabularnewline
\hline 
\textbf{y} & 5 & 0 & 6\tabularnewline
\hline 
\textbf{z} & \ensuremath{\infty} & \ensuremath{\infty} & \ensuremath{\infty}\tabularnewline
\hline 
\end{tabular}=> %
\begin{tabular}{|c|c|c|c|}
\hline 
 & \textbf{x} & \textbf{y} & \textbf{z}\tabularnewline
\hline 
\hline 
\textbf{x} & 0 & 5 & 2\tabularnewline
\hline 
\textbf{y} & 5 & 0 & 6\tabularnewline
\hline 
\textbf{z} & 2 & 6 & 0\tabularnewline
\hline 
\end{tabular}=> Änderung zu c(y,z)=8 =>%
\begin{tabular}{|c|c|c|c|}
\hline 
 & \textbf{x} & \textbf{y} & \textbf{z}\tabularnewline
\hline 
\hline 
\textbf{x} & 0 & 5 & 2\tabularnewline
\hline 
\textbf{y} & 5 & 0 & 8\tabularnewline
\hline 
\textbf{z} & 2 & 8 & 0\tabularnewline
\hline 
\end{tabular}=>%
\begin{tabular}{|c|c|c|c|}
\hline 
 & \textbf{x} & \textbf{y} & \textbf{z}\tabularnewline
\hline 
\hline 
\textbf{x} & 0 & 5 & 2\tabularnewline
\hline 
\textbf{y} & 5 & 0 & 7\tabularnewline
\hline 
\textbf{z} & 2 & 7 & 0\tabularnewline
\hline 
\end{tabular}=>%
\begin{tabular}{|c|c|c|c|}
\hline 
 & \textbf{x} & \textbf{y} & \textbf{z}\tabularnewline
\hline 
\hline 
\textbf{x} & 0 & 5 & 2\tabularnewline
\hline 
\textbf{y} & 5 & 0 & 7\tabularnewline
\hline 
\textbf{z} & 2 & 7 & 0\tabularnewline
\hline 
\end{tabular}

z

\begin{tabular}{|c|c|c|c|}
\hline 
 & \textbf{x} & \textbf{y} & \textbf{z}\tabularnewline
\hline 
\hline 
\textbf{x} & \ensuremath{\infty} & \ensuremath{\infty} & \ensuremath{\infty}\tabularnewline
\hline 
\textbf{y} & \ensuremath{\infty} & \ensuremath{\infty} & \ensuremath{\infty}\tabularnewline
\hline 
\textbf{z} & 2 & 6 & 0\tabularnewline
\hline 
\end{tabular}=> %
\begin{tabular}{|c|c|c|c|}
\hline 
 & \textbf{x} & \textbf{y} & \textbf{z}\tabularnewline
\hline 
\hline 
\textbf{x} & 0 & 5 & 2\tabularnewline
\hline 
\textbf{y} & 5 & 0 & 6\tabularnewline
\hline 
\textbf{z} & 2 & 6 & 0\tabularnewline
\hline 
\end{tabular}=> Änderung zu c(y,z)=8 =>%
\begin{tabular}{|c|c|c|c|}
\hline 
 & \textbf{x} & \textbf{y} & \textbf{z}\tabularnewline
\hline 
\hline 
\textbf{x} & 0 & 5 & 2\tabularnewline
\hline 
\textbf{y} & 5 & 0 & 8\tabularnewline
\hline 
\textbf{z} & 2 & 8 & 0\tabularnewline
\hline 
\end{tabular}=>%
\begin{tabular}{|c|c|c|c|}
\hline 
 & \textbf{x} & \textbf{y} & \textbf{z}\tabularnewline
\hline 
\hline 
\textbf{x} & 0 & 5 & 2\tabularnewline
\hline 
\textbf{y} & 5 & 0 & 7\tabularnewline
\hline 
\textbf{z} & 2 & 7 & 0\tabularnewline
\hline 
\end{tabular}=>%
\begin{tabular}{|c|c|c|c|}
\hline 
 & \textbf{x} & \textbf{y} & \textbf{z}\tabularnewline
\hline 
\hline 
\textbf{x} & 0 & 5 & 2\tabularnewline
\hline 
\textbf{y} & 5 & 0 & 7\tabularnewline
\hline 
\textbf{z} & 2 & 7 & 0\tabularnewline
\hline 
\end{tabular}
\end{comment}
\begin{comment}
\#14-12-5
\end{comment}

\aufgabe{Bonus, 3}Lesen Sie im Artikel \emph{Use Protection if Peering
Promiscuously}\footnote{\href{http://research.dyn.com/2014/11/use-protection-if-peering-promiscuously/}{http://research.dyn.com/2014/11/use-protection-if-peering-promiscuously/}}
der Dyn DNS Company die Abschnitte \emph{Recap: How this works} sowie
\emph{Conclusion} und beantworten Sie folgende Fragen:
\begin{itemize}
\item Was ist Peering und warum setzen es ISPs ein?%
\begin{comment}
1/2 P

Peering ist ein einvernehmliches Abkommen zweier mit einander verbundener
ISPs, Netzwerkverkehr (in beide Richtungen), der zwischen beiden ISPs
verläuft, kostenfrei weiterzuleiten.
\end{comment}
\item Was ist ein Routing Leak und was ist jeweils die Ursache der Routing
Leaks?%
\begin{comment}
1P

Route leaks sind ein Verstoß gegen das BGP und bestehen im falschen
Verlautbaren von Routing-Information an Nachbarn. Im Szenario 1 erfährt
ISP B von ISP A die Routing-Information zu ISP A's Kunden. Er gibt
diese fälschlicherweise an seine anderen Nachbarn unverändert als
direkte Routing-Informationen weiter (und verkürzt so die scheinbaren
Kosten von Routen zu A's Kunden über B). Damit wird Verkehr von B's
Nachbarn zu Kunden von A über B geleitet werden, obwohl dies nicht
unbedingt der kürzeste Weg ist. Im Szenario 2 gibt B umgekehrt die
Routing-Informationen von seinen Nachbarn direkt an A weiter. So sehen
für die A die Routen über B am günstigsten aus und ausgehender Verkehr
von A an Nachbarn von B wird über B geroutet werden.
\end{comment}
\item Was bedeuten die roten Dreiecke mit den Ausrufezeichen und was die
grünen bzw. violetten Pfeile in den animierten Diagrammen?%
\begin{comment}
1/2 P

Grüne Pfeile bedeuten BGP-Announcements, wobei gefälschte Announcements
(die gegen das Protokoll verstoßen) mit den roten Dreiecken gekennzeichnet
sind.

Violette Pfeile bezeichnen die Pfade des entsprechenden Internetverkehrs
aufgrund der Routing-Information.
\end{comment}
\item Was empfiehlt Dyn als Gegenmaßnahmen gegen Routing Leaks?%
\begin{comment}
1P
\begin{itemize}
\item Filter und Überwachung einsetzen, um sicherzustellen, dass keine gefälschten
Routen akzeptiert werden
\item Die Pfade überwachen, die der Verkehr nimmt, der durchs eigene Netzwerk
geht - u.a. mit Software aus dem Hause Dyn (,,use protection'')
- um festzustellen, ob ein Peer Routes leakt.
\item Peering vermeiden, falls es sich vom Volumen eh nicht lohnt
\end{itemize}
\end{comment}
\end{itemize}
\aufgabe{2}Erinnern Sie sich an das ARP-Protokoll (Vorlesung 08).
Was erwarten Sie beim Ausführen folgender Befehle? Sie können Ihre
These auch mit Wireshark überprüfen.
\begin{itemize}
\item \texttt{arp -d <IP-des-Gateways>} (benötigt i.d.R. Administratorrechte)%
\begin{comment}
1/2 P

Dieser Befehl löscht den Eintrag, der die MAC-Adresse zur IP des Gateways
zuordnet. Da jeder Verkehr ins weitere Internet über das Gateway geht,
wird die MAC-Adresse bald benötigt (viele Anwendungen im Hintergrund
nutzen das Internet). Daher wird es bald ein ARP-Request vom eigenen
Host geben, der vom Gateway beantwortet wird.
\end{comment}
\item \texttt{ping <Broadcast-Adresse-des-Subnetzes>} (bei Linux muss man
noch die Option \texttt{-b} hinzufügen)%
\begin{comment}
1/2 P

Je nach Layout des Subnetzes wird diese Ping-Nachricht an einige oder
alle Hosts des Subnetzes geschickt werden. Nicht alle erreichten Hosts,
die auf die ICMP-Nachricht antworten, kennen die MAC-Adresse unseres
Hosts und sie werden dementsprechend ARP-Requests (per Broadcast)
mit der Absender-IP des Ping verschicken. Der eigene Host wird diese
dann (per Unicast) beantworten.
\end{comment}
\item Recherchieren und erklären Sie \emph{ARP-Spoofing} und wozu es eingesetzt
wird.
\end{itemize}
\aufgabe{2}Betrachten Sie folgende Netzwerkkonfiguration. Die beiden
Zeilen neben einem Netzwerkinterface geben die IP-Adresse (oben) und
die MAC-Adresse (unten) an.

\begin{center}
\includegraphics[width=1\columnwidth,bb = 0 0 200 100, draft, type=eps]{MAC.png}
\par\end{center}
\begin{itemize}
\item Wird der Adapter von B die Rahmen verarbeiten, wenn A Tausende von
Rahmen an den Gateway-Router sendet, wobei jeder Rahmen an die LAN-Adresse
des Gateway-Routers adressiert ist? Falls dies zutrifft, wird der
Adapter von B die IP-Datagramme in diesen Rahmen an B's Netzwerkschicht
weiterleiten? Wie würden Ihre Antworten lauten, wenn A Rahmen mit
der Broadcast-Adresse des LAN senden würde?%
\begin{comment}
1P

B's Adapter wird die Rahmen lesen, aber nicht im Protokollstapel weiter
nach oben leiten (falls das Protokoll korrekt implementiert ist),
da die Adresse nicht mit der eigenen übereinstimmt. Im Fall der Broadcastadresse
gibt es die Datagramme an die IP-Schicht weiter (siehe obige Aufgabe
mit ping).
\end{comment}
\item Angenommen, A sendet ein Datagramm an F. Geben Sie die Quell- und
Ziel-MAC-Adresse des Rahmens an, der das IP-Datagramm verkapselt,
während

\begin{enumerate}
\item es von A an den linken Router bzw.
\item vom linken Router an den rechten Router bzw.
\item vom rechten Router an F
\end{enumerate}
geschickt wird. Geben Sie auch jeweils die IP-Adressen des verkapselten
Datagramms zu jedem der drei Zeitpunkte an.%
\begin{comment}
1/2 P
\begin{enumerate}
\item von A an den linken Router:

\begin{itemize}
\item Quell-MAC-Adresse: 00-00-00-00-00-00, Ziel-MAC-Adresse: 22-22-22-22-22-22
\item Quell-IP: 111.111.111.001, Destination IP: 133.333.333.003
\end{itemize}
\item vom linken Router an den rechten Router:

\begin{itemize}
\item Quell-MAC-Adresse: 33-33-33-33-33-33, Ziel-MAC-Adresse: 55-55-55-55-55-55
\item Quell-IP: 111.111.111.001, Destination IP: 133.333.333.003
\end{itemize}
\item vom rechten Router an F:

\begin{itemize}
\item Quell-MAC-Adresse: 88-88-88-88-88-88, Ziel-MAC-Adresse: 99-99-99-99-99-99
\item Quell-IP: 111.111.111.001, Destination IP: 133.333.333.003
\end{itemize}
\end{enumerate}
\end{comment}

\item Nehmen Sie jetzt an, dass der linke Router durch einen Switch ersetzt
wird und dass A, B, C, D und der rechte Router sternförmig mit diesem
Switch verbunden sind. Nehmen Sie an, A sendet erneut ein Datagramm
an F. Geben Sie die Quell- und Ziel-MAC-Adresse des Rahmens an, der
das IP-Datagramm verkapselt, während

\begin{enumerate}
\item es von A an den Switch bzw.
\item vom Switch an den rechten Router bzw.
\item vom rechten Router an F
\end{enumerate}
geschickt wird. Geben Sie auch jeweils die IP-Adressen des verkapselten
Datagramms zu jedem der drei Zeitpunkte an.%
\begin{comment}
1/2 P
\begin{enumerate}
\item von A an den Switch:

\begin{itemize}
\item Quell-MAC-Adresse: 00-00-00-00-00-00, Ziel-MAC-Adresse: 55-55-55-55-55-55
\item Quell-IP: 111.111.111.001, Destination IP: 133.333.333.003
\end{itemize}
\item vom Switch an den rechten Router:

\begin{itemize}
\item Quell-MAC-Adresse: 00-00-00-00-00-00, Ziel-MAC-Adresse: 55-55-55-55-55-55
\item Quell-IP: 111.111.111.001, Destination IP: 133.333.333.003
\end{itemize}
\item vom rechten Router an F:

\begin{itemize}
\item Quell-MAC-Adresse: 88-88-88-88-88-88, Ziel-MAC-Adresse: 99-99-99-99-99-99
\item Quell-IP: 111.111.111.001, Destination IP: 133.333.333.003
\end{itemize}
\end{enumerate}
\end{comment}

\end{itemize}
\aufgabe{Bonus, 2}Bei der Einwahl in unverschlüsselte WLANs, etwa
in Cafés und Hotels (oder WLANs wie UNI-WEBACCESS\footnote{\href{https://www.urz.uni-heidelberg.de/netz/laptop/wism-webauth.html}{https://www.urz.uni-heidelberg.de/netz/laptop/wism-webauth.html}}),
muss man zunächst die Geschäftsbedingungen akzeptieren, sich anmelden
oder über einen Bezahlservice einen zeitbeschränkten Internetzugang
erwerben. Dies geschieht heute oft über sogenannte ,,Captive Portals''\footnote{\href{http://www.golem.de/news/captive-portals-ein-workaround-der-bald-nicht-mehr-funktionieren-wird-1602-118963.html}{http://www.golem.de/news/captive-portals-ein-workaround-der-bald-nicht-mehr-funktionieren-wird-1602-118963.html}}.
\begin{itemize}
\item Erklären Sie in eigenen Worten, wie ein Captive Portal technisch implementiert
wird.
\item Geben Sie vier der Probleme an, die die HTTP-Arbeitsgruppe des IETF
im Kontext von Captive Portals\footnote{\href{https://github.com/httpwg/wiki/wiki/Captive-Portals}{https://github.com/httpwg/wiki/wiki/Captive-Portals}}
identifiziert hat, und erklären Sie sie in eigenen Worten.
\end{itemize}
\aufgabe{Bonus, 1}Lesen Sie die Geschichte der ,,500-Meilen-Email''\footnote{\href{http://www.ibiblio.org/harris/500milemail.html}{http://www.ibiblio.org/harris/500milemail.html}}.
Um weitere Details zu klären, gibt es ein FAQ\footnote{\href{http://www.ibiblio.org/harris/500milemail-faq.html}{http://www.ibiblio.org/harris/500milemail-faq.html}}
zur Geschichte. Beantworten Sie folgende Fragen:
\begin{description}
\item [{a)}] In welcher Schicht des TCP/IP-Stapels liegt die Fehlerursache?
Was war die Ursache? %
\begin{comment}
Die Fehlerursache liegt in der Anwendungsschicht, beim SMTP-Protokoll.
Die SMTP-Software war aktualisiert worden. Die Konfigurationsdatei
passte nicht zur neuen Version, wurde aber ohne einen Fehler anzuzeigen
eingelesen. Konfigurationsoptionen, die die neue Version nicht richtig
parsen konnten, wurden auf den Defaultwert 0 gesetzt. Insbesondere
wurde der Timeout-Wert zum Verbinden zu einem SMTP-Server auf 0 gesetzt.
Experimentell ließ sich feststellen, dass effektiv nach etwa 3ms ein
Verbindungsversuch abgebrochen wurde.
\end{comment}
\item [{b)}] Warum erwähnt der Autor, dass das Campusnetz nach seiner Erinnerung
keine Router einsetzte? Wieso gab es auch Orte, die weniger als 500
Meilen entfernt liegen und trotzdem nicht per Email erreichbar waren?%
\begin{comment}
Dass keine Router im Netzwerk vorhanden waren, führte dazu dass bis
zum Anschluss ans Internet keine Verarbeitungsverzögerung, Warteschlangenverzögerung
oder Übertragungsverzögerung auftraten. So kam bis zum entfernten
SMTP-Server wenig Verzögerung zur Ausbreitungsverzögerung hinzu.

Manche Standorte waren zwar geographisch näher als 500 Meilen, aber
bezüglich der Netzwerktopologie weiter entfernt. Hinzukommen kann
bei bestimmten Routen zusätzlicher Delay durch Middleboxes wie Firewalls.
\end{comment}
\begin{comment}
\#14-11-1
\end{comment}
\end{description}
\aufgabe{3}Erinnern Sie sich an \emph{Einen Tag im Leben einer Web-Anfrage.}
Angenommen, Sie schreiben Ihr eigenes Betriebssystem und wollen selbst
den TCP/IP-Netzwerkstapel implementieren (inklusive Firmware für Ihre
Ethernet-Netzwerkkarte). Zählen Sie (in der Reihenfolge der ersten
Verwendung) die Protokolle auf, die bei einem Aufruf einer Webseite
benutzt werden. Nehmen Sie dabei an, dass Sie Ihren Computer erst
unmittelbar vor diesem Aufruf an ein Netzwerk anschließen. Der Netzwerkadministrator
hat Ihnen gesagt, dass Sie keine feste IP-Adresse zugewiesen bekommen.

Geben Sie für jedes Protokoll an, wie viele einzelne Aktionen bezüglich
dieses Protokolls auf Ihrem Computer (mindestens) durchgeführt werden
müssen, bis der Aufruf durchgeführt ist. Welche Gegebenheiten oder
Ereignisse können dazu führen, dass mehr Schritte (auf Ihrem Computer)
unternommen werden müssen, und welche Protokolle betrifft dies?%
\begin{comment}
1. CSMA/CD: Um irgendwelche Rahmen zu verschicken, muss gelauscht
werden, ob die Ethernet-Leitung frei ist. Der erste Rahmen enthält
die DHCP-DISCOVER-Anfrage.

2. UDP: Die DHCP-Nachricht wird in UDP verpackt

3. IP: Die UDP-Nachricht wird in IP verpackt (mit Broadcast-Adresse)

4. DHCP: DISCOVER-Anfrage, um eine IP-Adresse anzufragen

5. ARP: Der Host kennt jetzt seine eigene IP, die seines Gateways/First-hop-Routers
und die des DNS-Servers. Jetzt will er erfahren, unter welcher MAC-Adresse
der Gateway zu erreichen ist. Oder, falls der DNS-Server im gleichen
Subnetz liegt, dessen MAC-Adresse.

6. DNS: Jetzt muss die URL aufgelöst werden.

7. TCP: Jetzt soll der erste Datenstrom geöffnet werden. Alle bisherigen
Nachrichten waren von organisatorischer Natur und benötigen nicht
die Transportschicht oder lediglich UDP.

8. HTTP: Nachdem alle Adressen klar sind und eine TCP-Verbindung besteht,
kann die HTTP-Anfrage gestellt werden.

HTTP: mind. 2 Aktionen: 1xRequest schicken, 1xResponse entgegennehmen
(möglicherweise mehr Nachrichten: inkompatible Version, Webseite verzogen/unbekannt/unerreichbar...
- dadurch natürlich auch mehr TCP- und IP-Aktionen)

TCP: mind. 5 Aktionen: 1xSYN schicken, 1xSYNACK entgegennehmen, 1xHTTP-Request
verpacken (gleichzeitig das SYNACK ack'en), 1xHTTP-Response entpacken
(angenommen, dass die Webseite klein genug ist), 1xACK für HTTP-Response
abschicken (möglicherweise mehr Nachrichten: Pakete gehen verloren
bei Verbindungsaufbau oder bei Stromübertragung, HTTP-Nachricht ist
länger als ein Paket...)

DHCP: mind. 4 Aktionen: 1xDHCPDISCOVER abschicken, 1xDHCPOFFER entgegennehmen,
1xDHCPREQUEST abschicken, um das Angebot anzunehmen, 1xDHCPACK entgegennehmen
(möglicherweise mehr Nachrichten: Falls es mehrere DHCP-Server gibt,
die auf das DISCOVER antworten, kommen evtl. mehrere DHCPOFFERs an;
falls Client oder Server aus einem Grund eine bestimmte Adresszuweisung
ablehnen; falls der User sich viel Zeit im Browser lässt, könnte die
Lease-Zeit ablaufen und ein DHCP-Refresh wird notwendig...)

UDP: mind. 6 Aktionen: je ein Paket pro DHCP- und DNS-Nachricht

IP: mind. 11 Aktionen: je ein Datagramm pro UDP- bzw. TCP-Paket (möglichweise
mehr Datagramme: falls die MTU auf einer Teilstrecke überschritten
wird, muss erneut gesendet werden, Datagramme können verloren gehen...)

ARP: mind. 2 Aktionen: 1xARP-Request an IP-Adresse des Gateway schicken,
1xARP-Reply entgegennehmen und in ARP-Cache eintragen (möglicherweise
mehr Aktionen: es können jederzeit ARP-Requests eingehen, da Teilnehmer
des Netzwerks diese broadcasten - ARP-Replies hingegen gehen nur gezielt
an deren Adressaten)

DNS: mind. 2 Aktionen: 1xDNS-Anfrage schicken, 1xDNS-Antwort entgegennehmen

CSMA/CD: mind. 11 Aktionen: alle 11+ IP-Datagramme müssen in Rahmen
verpackt werden und auf die Leitung gesendet werden; dabei achtet
der Netzwerkcontroller darauf, dass die MTU des ausgehenden Links
nicht überschritten wird (auf anderen Links später könnte dies allerdings
der Fall sein, was mehr Rahmen erforderlich machen kann - je nach
IP-Implementation auf den beteiligten Knoten; außerdem kann es zu
Kollisionen kommen, falls es sich nicht um geswitchtes Ethernet handelt)
\end{comment}
\begin{comment}
\#15-12-7
\end{comment}

\end{document}

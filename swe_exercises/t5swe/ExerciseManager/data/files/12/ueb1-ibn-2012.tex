\documentclass[a4paper,
headinclude,
footinclude,
11pt,
DIV14,
smallheadings]{scrartcl}

\usepackage[T1]{fontenc}
\usepackage[utf8]{inputenc}
\usepackage[ngerman]{babel}
\usepackage{xspace}
\usepackage{url}
\usepackage{caption}
\usepackage[autolanguage]{numprint}
\usepackage{booktabs}
\usepackage{tikz}
%\usepackage{libertine} 
\usepackage{ifthen}
\usepackage{paralist}
\usepackage{scrpage2}

\newcommand{\uebnr}{1}
\newcommand{\uebdatum}{18.04.2012}
\newcommand{\uebabgabe}{25.04.2011, 11:00 Uhr}


\setlength{\parindent}{0pt} 

\newcounter{uebnummer}
\setcounter{uebnummer}{0}

\newcommand{\aufgabe}[1]{
\stepcounter{uebnummer}
\section*{Aufgabe \arabic{uebnummer} \hfill (#1 \ifthenelse{\equal{#1}{1}}{Punkt}{Punkte})}
}

\pagestyle{scrheadings}

\ifoot{\footnotesize{\uebnr{}. Übung IBN -- SoSe 2012}}
\cfoot{}
\ofoot{\thepage}

\begin{document}

\newcommand{\befehl}[1]{"`\texttt{#1}"'\xspace}


Universität Heidelberg \hfill Lehrstuhl für Parallele und Verteilte Systeme (PVS)\\
Institut für Informatik\mbox{} \hfill Artur Andrzejak

\begin{center}
\begin{large}
\bfseries
\uebnr{}. Übung zur Vorlesung "`Betriebssysteme und Netzwerke"' \\
\end{large}
\vspace{2ex}

Ausgabedatum: \uebdatum  \qquad Abgabedatum: \uebabgabe
\end{center}
\hrule

\aufgabe{4}

Nennen Sie 4 Betriebssysteme Ihrer Wahl und geben Sie zu jedem folgende Informationen an:
\begin{compactitem}
\item das Jahr der ersten Veröffentlichung,
\item das Standarddateisystem,
\item die unterstützten Prozessorarchitekturen (CPUs),
\item ob das Betriebssystem einen monolithischen oder einen Mikrokernel besitzt,
\item und ob das Betriebssystem POSIX-Systemaufrufe bereitstellt.
\end{compactitem}

\aufgabe{2}
Beschreiben Sie in 2-3 Stichpunkten ein weiteres Beispiel für das Phänomen, dass ein Konzept der Computersysteme / Betriebssysteme oft verworfen wird, um Jahre später - ggf. in einem anderen Kontext - wiederaufzutauchen (siehe Vorlesung 1). Hinweis: mehr dazu finden Sie im Kapitel 1 ("`Einführung"') des Buches von Tanenbaum.

\aufgabe{2}

Was bedeutet der Begriff "`Spooling"' und welches Problem soll dieses Konzept lösen?

\aufgabe{2}
Definieren Sie den Begriff  "`Microcode"' (möglichst knapp, aber verständlich). Welche Vorteile und welche Nachteile hat dieser Ansatz (je 2-3 Stichpunkte)?  Nennen Sie mindestens zwei Familien von modernen Prozessoren, die dieses Konzept verwenden.


%http://en.wikipedia.org/wiki/Microcode
% Performance disadvantages were severe, and have let to RISC architectures. More details:
% Analysis shows complex instructions are rarely used, hence the machine resources devoted to them are largely wasted.
% Programming has largely moved away from assembly level, so it's no longer worthwhile to provide complex instructions for productivity reasons.
% The machine resources devoted to rarely-used complex instructions are better used for expediting performance of simpler, commonly-used instructions.
% Complex microcoded instructions requiring many, varying clock cycles are difficult to pipeline for increased performance.
% Simpler instruction sets allow direct execution by hardware, avoiding the performance penalty of microcoded execution.


\aufgabe{6}

Starten Sie eine Unix-Shell auf einem Ihnen zu Verfügung stehenden System (Mac OS X: Terminal, Windows: Cygwin, Linux: Terminal/Console). 
Mit dem Befehl \befehl{echo \$SHELL} können Sie feststellen, welche Shell gestartet wurde.
Sollte die Ausgabe nicht \befehl{/bin/bash} sein, starten sie eine Bash-Shell durch Eingabe des Kommandos \befehl{bash}.

Mit \befehl{set} bekommen Sie eine Übersicht über die in Ihrer Shell-Umgebung definierten Variablen.
Mit der Pipe \befehl{|} können Sie mehrere Befehle verketten. Mit Hilfe von \befehl{man} kann man Hilfe zu einem Befehl erhalten. Informieren Sie sich mittels \befehl{man} über folgende Befehle: \befehl{grep}, \befehl{ls}, \befehl{more}, \befehl{cat}, \befehl{head} (Beispiel: \befehl{man grep}).

\paragraph*{Teil 1 (2 Punkte)}
Welche Ausgabe produziert Ihr System bei Eingabe folgender Befehle?
\begin{compactitem}
\item \texttt{set | grep PATH}
\item \texttt{echo \$HOME}
\item \texttt{seq 1 10  | head -4}
\item \texttt{seq 1 50 | sort | head}
\item \texttt{alias myfancycommand='date'} \\
         \texttt{myfancycommand}
\end{compactitem}
Hinweis: Auf einem Mac ist statt \befehl{seq} der Befehl \befehl{jot} zu verwenden. Bsp. \befehl{seq 1 10} wird zu 
\befehl{jot 10 1 10}.

\paragraph*{Teil 2 (4 Punkte)}
Geben Sie eine Befehlskette an, um die ersten 5 Dateien eines Verzeichnisses auf der Kommandozeile auszugeben.




\end{document}

